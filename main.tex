\documentclass[a5paper,twosided,11pt,DIV=15,BCOR=0mm]{scrbook}
%
% hippopotato
\usepackage{localsettings}
\usepackage{cancel}
%
% lazyeqn - math symbols
\usepackage{./lazyeqn/lazyeqn}
\usepackage{hyperref}
%
\newcommand{\veps}{\vareps}
\newcommand{\ext}{\ensuremath{\textup{ext}}}
\newcommand{\inte}{\ensuremath{\textup{int}}}
\newcommand{\rhoz}{\tfrac{1}{\rho_0}}
%
\newcommand{\rtwo}{\UI\hspace{-0.07em}\UI}
\newcommand{\rthree}{\UI\hspace{-0.07em}\UI\hspace{-0.07em}\UI}

% \setcounter{secnumdepth}{3}
%
% \def\mybox#1{\fbox{%
%     \addtolength{\linewidth}{-2\fboxsep}%
%     \addtolength{\linewidth}{-2\fboxrule}%
%     \begin{minipage}[t]{\linewidth}%
%       % \vspace{-0.6em}%
%       #1%
%       % \vspace{0.5em}%
%     \end{minipage}%
%   }%
% }
% \def\myboxeqn#1{\fbox{%
%     \addtolength{\linewidth}{-2\fboxsep}%
%     \addtolength{\linewidth}{-2\fboxrule}%
%     \begin{minipage}[t]{\linewidth}%
%       \vspace{-0.6em}%
%       #1%
%       \vspace{-1em}%
%     \end{minipage}%
%   }%
% }
%
\raggedbottom
%
% \setlist{nolistsep,leftmargin=*}
%
% Fonts
%\usepackage[garamond]{mathdesign}
%\usepackage{times}
%\usepackage{pxfonts}%ccfonts,times,mathpazo,mathpple
\usepackage[T1]{fontenc}
%
% Search directory for image files
\graphicspath{{./img/}}
%
%
% Define BibTeX command
\def\BibTeX{{\rm B\kern-.05em{\sc i\kern-.025em b}\kern-.08em
    T\kern-.1667em\lower.7ex\hbox{E}\kern-.125emX}}
\title{Foundations of Single and Multiphasic Materials}
\author{Instructor: \\Digitized: H. Onur Solmaz, BSc}
%
% -----------------------------------------------------------------------
%
\begin{document}
%
\maketitle
\tableofcontents
\newpage
{\footnotesize
\subsection*{About C1: Continuum Mechanics}
Continuum-mechanical knowledge is the fundamental basis for the computation of
deformation processes of solid materials. Based on the methods of tensor
calculus, the lecture offers the following content:
%
\begin{itemize}
\item Vector and Tensor Algebra: symbols, spaces, products, specific
  tensors and definitions Vector and Tensor Analysis: functions of scalar-,
  vector- and tensor-valued variables, integral theorem (e. g., after Gauss or
  Stokes)
\item Foundations of Continuum Mechanics: kinematics and deformation, forces and
  stress concepts: Cauchy’s lemma and theorem, Cauchy, Kirchhoff and
  Piola-Kirchhoff stress tensors
\item Fundamental Balance Laws: master balance, axiomatic balance relations of
  mechanics (mass balance, momentum and angular momentum balances)
\item Related Balance Laws and Concepts: balance of mechanical energy, stress
  power and the concept of conjugate variables, d’Alembert’s principle and the
  principle of virtual work
\item Numerical Aspects of Continuum Mechanics: strong and weak formulation of
  the boundary-value problem
\item The Closure Problem of Mechanics: finite elasticity of solid mechanics (as
  an example), linearization of the field equations
\end{itemize}
\subsection*{About E2-01: Foundations of Single- and Multiphasic Materials}
The description of material behaviour under realistic 3-dimensional behaviour is
the fundamental basis for the computation of deformation processes of solid
materials and coupled deformation and flow processes in porous media. Based on
the methods of continuum mechanics, the lecture offers the following content:
%
\begin{itemize}
\item Thermodynamic Balance Laws: balance of energy, entropy inequality,
  thermodynamic potentials, application to finite thermoelasticity
\item Geometrically Linear Thermoelasticity: linearization of the finite
  problem, inversion of the linear law of thermoelasticity, determination of
  material parameters
\item Geometrically Linear Viscoelasticity: motivation and basic model rheology,
  the standard model of viscoelasticity (Poynting-Thomson model),
  Clausius-Planck inequality and internal dissipation, the viscoelastic solid
\item Geometrically Linear Elasto-Plasticity: motivation and basic model
  rheology, metal plasticity, generalised and geomaterials plasticity
\item Foundations of Multiphasic Materials: homogenisation, basic ideas and
  concepts
\item Hydraulics in Porous Materials: basic concepts, mechanical balance
  relations, constitutive equations, Darcy’s filter law
\end{itemize}
}
\chapter{Vector and Tensor Algebra}
\section{Symbols}
\subsection{Symbols with Single or Multiple Subscripts}
\begin{alignat}{3}
\nonumber&u_i         && \Rightarrow && u_1, u_2, u_3 \\
         &u_iv_k \quad&& \Rightarrow \quad&& u_1v_1, u_1v_2, u_1v_3, u_2v_1, u_2v_2, u_2v_3 \ldots \\
\nonumber&t_{ik}      && \Rightarrow && t_{11}, t_{12}, t_{13}, t_{21}, t_{22}, t_{23} \ldots
\end{alignat}
%
\subsection{Einstein's Summation Convention}
Wherever the same subscript occurs twice in a term, a summation over this
\emph{dummy} index is defined. Note that no index can occur more than two times.
%
\begin{equation}
  u_jv_j := \sum\limits_{j=1}^N u_jv_j = u_1v_1 + u_2v_2 + \ldots + u_Nv_N
\end{equation}

%
\subsection{Kronecker (Delta) Symbol}
$\delta_{ik}$ is a symbol with the following properties:
%
\begin{equation}
  \delta_{ik} = \left\{
    \begin{array}{l l}
      1 & \quad \text{if $i=k$}\\
      0 & \quad \text{if $i\neq k$}
    \end{array} \right.
\end{equation}
%
If $\delta_{ik}$ is multiplied with another quantity and if there is a double
subscript, then $\delta_{ik}$ disappears, the double subscript is dropped and
the \emph{free} subscript remains.
%
\section{Notions and Definitions of Vector Algebra}
\subsection{Addition of Vectors}
For $\cbr{\Vu,\Vv,\Vw}\in v^3$, the following relations hold
%
\subsection{Multiplication of Vectors with Scalar Quantities}
%
\subsubsection{Applications in the Affine Vector Space}
%
\paragraph{Linear Dependency of Vectors}
%
\paragraph{Basis Vectors in $v^3$}
%
\subparagraph{Choice of Special Basis}
%
\subparagraph{Representation of $v$}
%
\begin{remark}[An Arbitrary Basis System]

\end{remark}
%
\subparagraph{Representation of $v$ in Different Basis Systems}
%
\subsection{Scalar Product of Vectors (Inner Product)}
%
\begin{remark}

\end{remark}
\subsubsection{Applications in the Proper Euclidean Space}
\paragraph{Square and Norm of Vectors}
\paragraph{Angle between Vectors}
\paragraph{Scalar Product of Basis Vectors}
\paragraph{Scalar Product of Vectors}

\subsection{Vector Product or Cross Product of Vectors (Outer Product)}
\subsubsection{Consequences and Applications of the Vector Product}
\paragraph{The Vector Product in an Orthonormal Frame (Basis)}
\paragraph{Norm (Value) of the Vector Product}
\paragraph{Scalar Triple Product}

\section{Notions and Definitions of Tensor Algebra}

\subsection{Introduction of the Concept of a Tensor}
\subsection{Tensor Product and Dyadic Product Space}

\subsubsection{Basis Notation of a Simple Tensor}
\subsubsection{Introduction of Arbitrary Tensors $\VT\in v^3 \dyc v^3$}

\subsection{Basic Rules of Tensor Algebra}

\subsubsection{Multiplication of Tensors by Scalars}

\subsubsection{Suppl. 2.2(b)}
\subsubsection{Linear Mapping of Vectors by Tensors}
\subsubsection{Scalar Product of Tensors}
\subsubsection{Tensor Product of Tensors}

\section{Specific Tensors and Operations}

\subsection{Definition of Specific Tensors of 2nd Order}
\subsubsection{Transposed Tensor}
\subsubsection{Symmetric and Skew-symmetric Tensors}
\subsubsection{Inverse Tensors}
\subsubsection{Orthogonal Tensors}
\subsubsection{Trace of a Tensor}

\subsection{High Order Tensors}

\paragraph{General Linear Mapping}

\subsubsection{Fundamental Tensors of 4th Order}
\begin{boxenv}[Properties of 4th Order Fundamental Tensors]
  \begin{enumerate}
  \item Identity Map
  \item Transposing Map
  \item Tracing Map
  \end{enumerate}
\end{boxenv}
\subsubsection{Specific 4th Order Tensors}


\subsection{Fundamental Tensors of 3rd Order}

\subsection{The Axial Vector and the Vector Product of Tensors}

\subsubsection{The Vector Product of Tensors}

\subsubsection{The Outer Tensor Product of Vectors and Tensors}

\subsection{Euler-Rodriguez Tensor of Spatial Rotation}

\subsection{The Outer Tensor Product of Tensors}

\subsubsection{The Cofactor and the Determinant}

\subsubsection{The Inverse Tensor}

\chapter{Fundamentals of Vector and Tensor Analysis}

\section{The Idea of Functions}

\section{Functions of Scalar Variables}

\subsection{Vector-valued Functions of a Single Variable}

\subsection{Vector-valued Functions of Several Variables}

\subsection{Tensor-valued Functions of One or Several Variables}

\subsection{Derivatives of Products of Functions}

\section{Functions of Vectorial and Tensorial Variables}

\subsection{The Gradient Operator}

\subsection{Functions of Arbitrary Vectorial and Tensorial Variables}

\subsection{Specific Operators}

\subsubsection{The Divergence Operator}

\subsubsection{The Rotation Operator}

\subsubsection{The Laplace Operator}

\section{Integral Theorems}


\chapter{Kinematics}
%
\section{Body, Placement, Motion}

\subsection{The Material Body in Different Configurations}

\subsection{The Placement Function}

\subsection{The Motion of a Body}

\subsection{Displacement, Velocity and Acceleration}

\subsection{The Material Time Derivative}


\section{The Deformation Gradient}

\subsection{Transformation of Line Elements}

\subsection{Investigation of the Initial Conditions}

\subsection{Transformation of Area Elements}

\subsection{Transformation of Volume Elements}

\begin{remark}[Time Derivative of the Volume Map]
  First, the following expression needs to be clarified
  \begin{equation}
    \VF\invtra \dtp\dot{\VF} = \VF\invtra\cdot\VL\VF
    = \VF\invtra\VF\tra\dtp\VL = \VI\dtp\VL =\div\dot{\Vx} \label{eq:detfderiv1}
  \end{equation}
  Then, the time derivative of the volume map is expanded as
  \begin{align}
\nonumber  \ddt[\det\VF] &= (\det\VF)\dot{}= \deriv{\det\VF}{\VF}\dtp\dot{\VF}=\overset{+}{\VF}\dtp\dot{\VF}\\
    &= \det\VF \,\underbrace{\VF\invtra\dtp\dot{\VF}}_\text{from \eqref{eq:detfderiv1}}
      = \det\VF\div\dot{\Vx} \label{eq:detfderiv2}
  \end{align}
\end{remark}

\section{Deformation and Strain Measures}

\subsection{The Polar Decomposition Theory}

\subsection{Deformation and Strain Tensors}

\subsection{Deformation and Strain using the Displacement Vector}

\section{Deformation and Strain Rates}

\subsection{Material and Spatial Velocity Gradient}

\subsection{Rate of Deformation and Rate of Rotation Tensors (Spin)}

\subsection{Rate of Deformation and Gyration Tensor}

\subsection{Right and Left Deformation Rates}

\subsection{Velocities of Inverse Measures}

\subsection{Green-Lagrangian and Almansian Strain Rates}

\subsection{Karni-Reiner Strain Rates}


\chapter{Stress Concepts}

\section{Volume and Surface Forces}

\section{Stress Tensors}

\subsection{Cauchy's Lemma}

\subsection{Cauchy's Theorem}

\subsection{Stress Tensors in Geometrically Finite Theories}

\subsection{Stress Tensors in Geometrically Linear Theories}
%
\chapter{Fundamental Balance Law, a.k.a. Master Balance}
The master balance relation exhibits the frame of all balance relations in
continuum mechanics. Generally, there are volume specific scalar $\Psi$ and
vectorial $\BPsi$ mechanical quantities to be balanced.
%
\\Here, scalar quantities $\Psi$ include the mass density $\rho$, internal energy
$\rho\veps$ and entropy $\rho\eta$. Vectorial quantities $\BPsi$ include the
momentum $\Vl$ and moment of momentum $\Vh_{(B)}$ where $B$ is an arbitrary but
spatially finite point.

\section{Global Representation of the Master Balance}
\begin{subequations}
  \begin{alignat}{4}
  &  \ddt\int_\CV\Psi\dv &&= \int_\CS\Bvarphi\dtp\Vn\da &&+ \int_\CV\sigma\dv &&+ \int_\CV\hat{\Psi}\dv\label{eq:globalbalance1}\\
  &  \ddt\int_\CV\BPsi\dv &&= \int_\CS\BPhi\Vn\da &&+ \int_\CV\Bsigma\dv &&+ \int_\CV\hat{\BPsi}\dv\label{eq:globalbalance2}
  \end{alignat}
\end{subequations}
%
Therein, the following field functions $\rbdot=\rbdot(\Vx,t)$ are defined.
%
\begin{description}
\item[$\Psi$, $\BPsi$] volume specific mechanical quantities,
\item[$\Bvarphi\dtp\Vn$, $\BPhi\Vn$] efflux of the mechanical quantity through the
  surface $\CS$ (action at the vicinity)
\item[$\sigma$, $\Bsigma$] supply of the mechanical quantity (action from the
  distance)
\item[$\hat{\Psi}$, $\hat{\BPsi}$] production of the mechanical quantity
  (exchange of $\CB$ with the environment of $\CB$)
\end{description}

\section{Local Representation of the Master Balance}
The goal is to transform the global balance which is valid for the overall body
$\CB$ into a local balance valid for each material point $P$.

\subsection{Scalar-valued Mechanical Quantities}
Investigation of the left hand side of \eqref{eq:globalbalance1} leads to
%
\begin{equation}
  \ddt\int_\CV\Psi\dv=\ddt\int_\CV\Psi\det\VF\dV=\int_\CV\ddt[\Psi\det\VF]\dV
\end{equation}
%
which using chain rule and \eqref{eq:detfderiv2} yields
%
\begin{equation}
  \ddt\int_\CV\Psi\dv=\ddt\int_\CV(\dot{\Psi}\det\VF+\Psi\div\dot{\Vx}\det\VF)\dV
  = \int_\CV(\dot{\Psi}+\Psi\div\dot{\Vx})\dv
\end{equation}
%
Investigating the first term of the right hand side of \eqref{eq:globalbalance1}
leads to
%
\begin{equation}
  \int_\CS\Bvarphi\dtp\Vn\da = \int_\CS\Bvarphi\dtp\dif\Va = \int_\CV\div\Bvarphi\dv
\end{equation}
%
Therein, the Gauss Integral Theorem from supplement 3.4(b) was used. Insertion
of these results back into \eqref{eq:globalbalance1} yields
%
\begin{equation}
  \int_\CV(\dot{\Psi}+\Psi\div\dot{\Vx})\dv = \int_\CV(\div\Bvarphi+\sigma+\hat{\Psi})\dv
\end{equation}
%
In case the integrands of this are steady and sufficiently differentiable, the
local representation of \eqref{eq:globalbalance1} is rendered as
%
\begin{equation}
  \dot{\Psi}+\Psi\div\dot{\Vx} = \div\Bvarphi+\sigma+\hat{\Psi}
\end{equation}

\subsection{Vector-valued Mechanical Quantities}
Analogously to the above procedure, one obtains
%
\begin{equation}
  \dot{\BPsi}+\BPsi\div\dot{\Vx} = \div \BPhi + \Bsigma + \hat{\BPsi}
\end{equation}
%
\chapter{Axiomatic Balance Relations of Mechanics}

\section{Balance of Mass}

\section{Balance of (Linear) Momentum}

\section{Balance of Moment of Momentum (Angular Momentum)}

\section{Balance of Mechanical Energy}

\chapter{Stress Power and Conjugate Variables}

\section{Volume Specific Stress Power}

\subsection{Alternative Representations of the Specific Stress Power}
\subsubsection{Spatial and Referential Representations}

\section{The Concept of Conjugate Variables}
\subsection{The 1D Geometrically Linear Case}
\subsection{The 3D  General Case}

\subsubsection{In the Reference Frame}
\subsubsection{In the Spatial Frame}
\section{Balance of Mechanical Energy in Conservative Systems}
\section{Internal Stresses in Conservative Systems}
\section{External Forces in Conservative Systems}

\chapter{d'Alembert's Principle and the Principle of Virtual Work}
\section{d'Alembert's Principle}

\chapter{Numerical Aspects of Continuum Mechanics}


\chapter{Closure Problem}
%
%
\rule{0.3\linewidth}{0.25pt}
End of semester 1
%
\chapter{Axiomatic Balance Relations of Thermodynamics}
\section{Balance of Energy}
%
\begin{equation}
  \ddt \sbr{E(\CB,t)+K(\CB,t)} = P_{\ext}(\CB,t)+Q_{\ext}(\CB,t)
\end{equation}
\begin{equation}
 \rho\dot{\veps} = \VT\cdot\VL - \div \Vq + \rho r
\end{equation}
%
\section{Balance of Entropy}
\begin{alignat*}{3}
  & \eta(\CP,t) && = \eta(\Vx, t) && \textrm{: mass specific entropy}\\
  & H(\CB,t) && = \int_\CV\rho\eta\dv\,\, && \textrm{: entropy of }\CB
\end{alignat*}
%
\begin{equation}
  \dot{H} = \frac{Q}{\theta}
\end{equation}
Processes are
\begin{itemize*}
\item reversible, if $\int \frac{Q}{\theta} \dt = 0$
\item irreversible, if $\int \frac{Q}{\theta} \dt > 0$
\end{itemize*}
\begin{align}
  S_\ext(\CB,t) &= \int_\CS\Bphi_\eta\cdot\Vn\da + \int_\CV\sigma_\eta\dv \\
  S_\inte (\CB,t) &= \int_\CV\hat{\eta}\dv
\end{align}
\begin{alignat*}{3}
  & \Bphi_\eta && := -\frac{1}{\theta}\Vq && \textrm{: heat influx/temperature}\\
  & \sigma_\eta && := \frac{\rho r}{\theta} && \textrm{: heat supply/temperature}\\
  & \hat{\eta}(\CP,t) && := \hat{\eta}(\Vx,t)\quad && \textrm{: volume specific
    entropy production}
\end{alignat*}
%
Global form:
\begin{equation}
    \ddt H(\CB,t) = S_\ext(\CB,t) + S_\inte(\CB,t)
\end{equation}
Local form is obtained by substituting the variables
\begin{equation}
 \rho \dot{\eta} = -\div\rbr{\frac{\Vq}{\theta}} + \frac{\rho r}{\theta} + \hat{\eta}
\end{equation}
2\textsuperscript{nd} Law dictates
\begin{equation}
\hat{\eta} \geq 0  \Rightarrow  \rho \dot{\eta} +\div\rbr{\frac{\Vq}{\theta}} -
  \frac{\rho r}{\theta} \geq 0
\end{equation}
After substitution of energy balance and Legendre transformation:
\begin{equation}
-\rho\rbr{\dot{\Psi} + \dot{\theta}\eta} + \VT\cdot\VL -
  \frac{1}{\theta}\Vq\cdot\grad\theta \geq 0
\end{equation}
%
\section{Thermodynamic Potentials}
%
\subsection{Internal Energy $\veps(\CP,t)$}
\begin{equation}
  \veps(\CP,t) = \veps(\Vx,t) = \veps(
    \underbrace{\VE(\Vx,t)}_\text{strain},
    \underbrace{\eta(\Vx,t)}_\text{entropy})
\end{equation}
Consequence:
\begin{equation}
  \dot{\veps} = \partd{\veps}{\VE}\cdot\dot{\VE}
  + \partd{\veps}{\eta}\dot{\eta}
  = \rhoz\VS\cdot\dot{\VE} + \theta\dot{\eta}
\end{equation}
  where $\rhoz\VS = \partd{\veps}{\VE}$ and $\theta = \partd{\veps}{\eta}$.
\subsubsection{Relation between  $\veps(\VE,\eta)$, the elastic potential $u(\VE)$ and the
internal stress power $w(\Vx,t)$}
\begin{equation}
  u(\VE) = \rho_0 \veps(\VE,\eta)\evat_{\eta=\textrm{const.}}
\end{equation}
The stress:
\begin{align}
  \VS = \deriv{u}{\VE} &= \partd{(\rho_0\veps|_{\eta=\textrm{const.}})}{\VE} =
  \rho_0\partd{(\veps|_{\eta=\textrm{const.}})}{\VE}\\
  &\Rightarrow \rhoz\VS = \partd{(\veps|_{\eta=\textrm{const.}})}{\VE} = \partd{\veps}{\VE}
\end{align}
Stress power:
\begin{align}
\dot{u}(\VE) &= \linedot{\rho_0 \veps(\VE,\eta)|_{\eta=\textrm{const.}}} \\
& = w(\Vx,t) = \VS\cdot\dot{\VE} = \deriv{u}{\VE}\cdot\dot{\VE}
  = \rho_0\partd{\veps}{\VE}\cdot\dot{\VE}
\end{align}
Conjugate pairs: $\cbr{\rhoz\VS,\VE}$ and $\cbr{\theta, \eta}$.
%
\subsection{Helmholtz Free Energy $\Psi(\CP,t)$}
Legendre transformation:
\begin{equation}
  \Psi(\VE,\theta) = \veps(\VE,\eta) - \theta\eta
\end{equation}
Consequence:
\begin{equation}
  \dot{\Psi}(\VE,\theta)
  = \partd{\Psi}{\VE}\cdot\dot{\VE} + \partd{\Psi}{\theta}\dot{\theta}
  = \rhoz\VS\cdot\dot{\VE} - \eta\dot{\theta}
\end{equation}
where $\rhoz\VS = \partd{\veps}{\VE}  = \partd{\Psi}{\VE}$ and
  $\theta = \partd{\veps}{\eta} \Leftrightarrow \eta = -\partd{\Psi}{\theta}$
\subsection{Enthalpy $\zeta(\CP,t)$}
Legendre transformation:
\begin{equation}
  \zeta(\rhoz\VS,\eta) = \veps(\VE,\eta) - \rhoz\VS\cdot\VE
\end{equation}
Consequence:
\begin{equation}
\dot{\zeta}(\rhoz\VS,\eta) = \partd{\zeta}{\rhoz\VS}\cdot(\rhoz\VS)\dot{}
+\partd{\zeta}{\eta}\dot{\eta}
= - \VE \cdot \rhoz\dot{\VS}+\theta\dot{\eta}
\end{equation}
where $\theta=\partd{\veps}{\eta}=\partd{\zeta}{\eta}$ and
$\VE=-\rho_0\partd{\zeta}{\VS}\Leftrightarrow\rhoz\VS=\partd{\veps}{\VE}$
\subsection{Gibbs' Free Enthalpy $\xi(\CP,t)$}
Legendre transformation:
\begin{equation}
  \xi(\rhoz\VS,\theta)=\Psi(\VE,\theta)-\rhoz\VS\cdot\VE
\end{equation}
Consequence:
\begin{equation}
\textstyle
    \dot{\xi}(\rhoz\VS,\theta) =
    \partd{\xi}{\rhoz\VS}\cdot\linedot{\rhoz\VS}+\partd{\xi}{\theta}\dot{\theta}
    =-\VE\cdot\rhoz\dot{\VS}-\eta\dot{\theta}
\end{equation}
where $\eta=-\partd{\Psi}{\theta} = -\partd{\xi}{\theta}$ and
$\VE=-\rho_0\partd{\xi}{\VS}\Leftrightarrow\rhoz\VS=\partd{\Psi}{\VE}$.
\subsection{Relationships Between the Potentials}
\begin{figure}[htbp]
  \centering
  \begin{alignat*}{3}
    &\boxed{\veps                }      && \Rightarrow -\rhoz\VS\cdot\VE      &\boxed{\zeta=\veps-\rhoz\VS\cdot\VE}\\
    &      \Uparrow +\theta\eta         &&                                    & \Downarrow -\theta\eta \\
    &\boxed{\Psi=\veps-\theta\eta} \quad&& \Leftarrow +\rhoz\VS\cdot\VE \quad&
    \boxed{\xi=\veps-\theta\eta-\rhoz\VS\cdot\VE}
  \end{alignat*}
  \caption{Legendre transformations between the potentials}
  \label{fig:legendre}
\end{figure}
% \begin{tabular}{l l l}
%   $\veps$ & $\Rightarrow -\rhoz\VS\cdot\VE$ &$\zeta=\veps-\rhoz\VS\cdot\VE$\vspace{0.5em}\\
%   $\Uparrow +\theta\eta$ & & $\Downarrow -\theta\eta$ \\[0.5em]
%   $\Psi=\veps-\theta\eta$ & $\Leftarrow +\rhoz\VS\cdot\VE$ & $\xi=\veps-\theta\eta-\rhoz\VS\cdot\VE$\\
% \end{tabular}
Identities for thermodynamic potentials:
\begin{itemize}
\item Motivated by the pair $\cbr{\rhoz\VS,\VE}$:
  \begin{equation}
    \veps-\zeta=\Psi-\xi=\rhoz\VS\cdot\VE
    \Leftrightarrow (\veps-\zeta)-(\Psi-\xi) = 0
  \end{equation}
\item  Motivated by the pair $\cbr{\eta,\theta}$:
  \begin{equation}
    \veps-\Psi=\zeta-\xi=\theta\eta
    \Leftrightarrow (\veps-\Psi)-(\zeta-\xi) = 0
  \end{equation}
\end{itemize}
%
%%%%%%%%%%%%%%%%%%%%%%%%%%%%%%%%%%%%%%%%%%%%%%%%%%
\chapter{Finite Strain Thermoelasticity}
Conjugate pairs $\cbr{\rhoz\VS,\VE}$ and $\cbr{\eta,\theta}$ are used.
%
\begin{subequations}
  \begin{alignat}{2}
    &  \text{momentum balance:  }&&\rho\ddot{\Vx}=\div \VT + \rho\Vb\\
    &  \text{energy balance:    }&&\rho\dot{\veps}=\VT\cdot\VL-\div\Vq+\rho r\\
    & \text{entropy inequality:}&&\textstyle-\rho(\dot{\Psi} + \dot{\theta}\eta)
    + \VT\cdot\VL - \frac{1}{\theta}\Vq\cdot\grad\theta \geq 0
  \end{alignat}
\end{subequations}
%
Description is fully Lagrangian with a solid reference configuration. Above
yields
\begin{subequations}
  \begin{alignat}{2}
    &\text{momentum balance:  }&&\rho_0\ddot{\Vx}=\Div(\VF\VS) + \rho_0\Vb\\
    &\text{energy balance:    }&&\rho_0\dot{\veps}=\VS\cdot\dot{\VE}-\Div\Vq_0+\rho_0 r\\
    &\text{entropy inequality:}&&\textstyle-\rho_0(\dot{\Psi} + \dot{\theta}\eta) +
    \VS\cdot\dot{\VE} - \frac{1}{\theta}\Vq_0\cdot\Grad\theta \geq 0
  \end{alignat}
\end{subequations}
%
Used identities:
\begin{align*}
 \rho&=\rho_0\rbr{\det\VF}^{-1}\\
 \div\VT&=(\det\VF)^{-1}\Div(\VF\VS)\\
 \VT\cdot\VL&=(\det\VF)^{-1}\VS\cdot\dot{\VE}\\
 \Vq&=(\det\VF)^{-1}\VF\Vq_0\\
 \div\Vq&=(\dev\VF)^{-1}\Div \Vq_0
\end{align*}
\section{Basic Principles}
\subsection{Determinism [Noll, 1958]}
\subsection{Equipresence [Truesdell, 1949]}
\subsection{Local Action [Noll, 1958]}
\subsection{Material Frame Indifference [Noll,1955]}
%
\section{Evolution of the Entropy Inequality (Dissipation)}
\begin{multline}
  - \rho_0\rbr{\eta+\partd{\Psi}{\theta}}\dot{\theta}
  + \rbr{\VS-\rho_0\partd{\Psi}{\VE}}\cdot\dot{\VE}\\
  - \rho_0\partd{\Psi}{\Grad\theta}\cdot\Grad\theta
  - \frac{1}{\theta}\Vq_0\cdot\Grad\theta \geq 0
\end{multline}
For arbitrary admissible processes:
\begin{subequations}
  \begin{align}
    \eta&=-\partd{\Psi}{\theta}\\
    \VS&=\rho_0\partd{\Psi}{\VE}\\
    \Vzero &= \partd{\Psi}{\Grad\theta}
  \end{align}
\end{subequations}
leading to the dissipation inequality:
%
\begin{equation}
  \CD := -\frac{1}{\theta}\Vq_o\cdot\Grad\theta \geq 0
\end{equation}
%
Satisfied with Fourier's law:
\begin{equation}
  \Vq_0=-k\Grad\theta
\end{equation}
%
\section{Specific Cases and Resulting Simplifications}
\subsection{Rigid Body Motion}
R.B.M. function:
\begin{equation}
  \Vx(\VX,t)=\VQ(t)\VX + \VK(t) \label{eq:rbm0}
\end{equation}
where $\VQ$ is orthogonal: $\VQ^{-1}=\VQ\tra$.
%
Deformation gradient: $\VF=\Grad\Vx=\VQ(t)$, $\dot{\VF}=\dot{\VQ}$
%
First, show that
\begin{equation}
  \VL\tra=(\dot{\VF}\VF^{-1})\tra=\VF\tra[-]\dot{\VF}\tra=\VQ\tra[-]\dot{\VQ}\tra
  =\VQ\dot{\VQ}^{-1}=\VF\dot{\VF}^{-1}\label{eq:rbm1}
\end{equation}
%
Secondly, continue from the identity $\VI=\VF\VF^{-1}$.
Its derivative becomes, using (\ref{eq:rbm1})
\begin{equation}
  \dot{\VI} = \underbrace{\dot{\VF}\VF^{-1}}_\text{$\VL$}
  + \underbrace{\VF\linedot{\VF^{-1}}}_\text{$\VL\tra$} = 0 \label{eq:rbm2}
\end{equation}
%
Hence
\begin{equation}
  \VL + \VL\tra = 0\quad\textup{\underline{with r.b.m.!!!}} \label{eq:rbm3}
\end{equation}
and this identity can help reduce many formulations. For example
$\VD$ was defined as $\VD=\sym(\VL)=\half(\VL+\VL\tra)$. This means
%
\begin{equation}
\VD=\Vzero \eqand \UI_D=\rtwo_D=0 \label{eq:rbm4}
\end{equation}
%
\subsection{Incompressibility}
%
The amount of mass is constant. Incompressibility means that the volume also
does not change, hence density remains constant:
\begin{equation}
  \rho:\textup{constant} \quadd{\rightarrow} \dot{\rho}=0
\end{equation}
%
The mass balance equation becomes
\begin{equation}
\cancelto{0}{\dot{\rho}}+\rho\,\div\dot{\Vx}= 0 \quadd{\Rightarrow}
\div\dot{\Vx} = \VL\cdot\VI = \VD\cdot\VI = 0
\end{equation}
%
\subsection{Adiabatic Process}
\begin{equation}
  r=0 \eqand \Vq=\Vzero
\end{equation}
Thus the energy balance becomes
%
\begin{equation}
  \rho\dot{\veps} = \VT\cdot\VL
\end{equation}
%
\subsection{Quasi-static Conditions}
\begin{equation}
  \deriv{K}{t} = 0
\end{equation}
\subsection{Isothermal Process}
\begin{equation}
  \dot{\theta} = 0
\end{equation}
\subsection{Isentropic Process}
\begin{equation}
  \dot{\eta} = 0
\end{equation}
\section{Examples of Constitutive Laws}
\subsection{Nonlinear Free Energy -- Extension of Simo--Pister Law}
\begin{align}
  \nonumber \rho_0\Psi(\VE,\theta)&\rightarrow\rho_0\Psi(\UI_C,\rthree_c,\theta)\\
  \nonumber &=\half\mu(\UI_C-3)+(m\Delta\theta-\mu)\ln(\rthree_c)^{1/2} \\
  & + \half\lambda[\ln(\rthree_c)^{1/2}]^2-
    \rho_0 c_v(\theta\ln\tfrac{\theta}{\theta_0}-\Delta\theta)
\end{align}
%
$\mu,\lambda$:~Lam\`e constants,
$c_v$:~specific heat at constant volume,
$m$:~stress-temperature modulus ($m=-(2\mu+3\lambda)\,\alpha = -3\kappa\alpha$, $\alpha$:~coeff of
thermal expansion, $\kappa$:~compression modulus)
Stress becomes
\begin{equation}
  \textstyle\VS=\rho_0\partd{\Psi}{\VE} = 2\rho_0\partd{\Psi}{\VC}
  = 2\rho_0 (\partd{\Psi}{\UI_C}\VI+\det\VC\partd{\Psi}{\rthree_C}\VC^{-1})
\end{equation}
Eventually
\begin{equation}
  \VS = 2\mu\,\VK^R+\lambda\ln(\det\VF)\,\VC^{-1}+m\Div\Grad\theta\,\VC^{-1}
\end{equation}
%
Entropy becomes
\begin{equation}
  \rho_0\theta\dot{\eta} = \rho_0c_v\dot{\theta}-m\theta(\dot{\VE}\cdot\VC^{-1})
\end{equation}
%
Energy becomes
\begin{equation}
  \rho_oc_v\dot{\theta} = m\theta\,(\dot{\VE}\cdot\VC^{-1})
  -\Div\Vq_0 + \rho_0\,r\label{eq:simoenergy}
\end{equation}
(\ref{eq:simoenergy}) is called the equation of thermal conduction.
%
An \underline{adiabatic system} ($-\Div\Vq_0 + \rho_0\,r=0$) yields
\begin{equation}
  \rho_oc_v\dot{\theta} = m\theta\,(\dot{\VE}\cdot\VC^{-1}) \quad\textup{or}\quad
  \theta\dot{\eta}=0
\end{equation}
For a hyperelastic material ($m=0$), substituting $\Vq_0$ from previously defined Fourier's Law:
\begin{equation}
  c_v\dot{\theta} = \frac{k}{\rho_0}\Div\Grad\theta+r
\end{equation}
Momentum balance: is not reduced
\begin{equation}
  \rho_0\ddot{\Vx}=\Div(\VF\VS(\VE,\theta)) + \rho_0\Vb\label{eq:simomomentum}
\end{equation}
\subsubsection{Coupled problem of thermoelasticity:} Equations (\ref{eq:simoenergy})
(equation of thermal conduction) and (\ref{eq:simomomentum}) (momentum balance)
together form the coupled problem of thermoelasticity.
%
The coupling results from the stress temperature modulus through
$\VS=\ldots+m\Div\Grad\theta\,\VC^{-1}$
%
This thermomechanical coupling is a weak coupling. Operator splitting methods
can be applied in numerical computations.
\subsection{Flory's Law}
Compressible, isotropic, finite elasticity law. In actual configuration:
\begin{equation}
  \Btau=2\mu(\det\VF)^{-2/3}\VK^D + k \ln(\det\VF)\VI
\end{equation}
In reference configuration:
\begin{equation}
  \VS=\mu(\det\VF)^{-2/3}\, \VI+\sbr{k\ln(\det\VF)-
    \third\mu(\det\VF)^{-2/3}(\VF\cdot\VF)}\VC^{-1}
\end{equation}
%
\subsection{Ideal Gas}
%
\subsection{Newtonian Fluid}
\begin{equation}
  \VT=-p\,\VI+\VT_E \quad \textrm{with} \quad
  \VT_E=2\mu\,\VD+\lambda(\VD\cdot\VI)\VI
\end{equation}
Energy balance for this:
\begin{equation}
  \rho\,\dot{\veps} = -p\, (\VI\cdot\VD)+2\mu\, \VD\cdot\VD
  +\lambda\,(\VD\cdot\VI)^2 + k\div\grad\theta+\rho\,r
\end{equation}
written in terms of invariants, utilizing $\VD\cdot\VD=\UI_D^2-2\rtwo_D$:
\begin{equation}
  \rho\,\dot{\veps}=-p\,\UI_D+(2\mu+\lambda)\,\UI_D^2 - 4\mu\,\rtwo_D
  + k\div\grad\theta + \rho\,r
\end{equation}
%
\subsubsection{Simplification: \emph{R.B.M} and \emph{only-temperature-dependent}
  internal energy}
%
From \eqref{eq:rbm4},
\begin{equation}
  \rho\,\dot{\veps} = k \div\grad\theta + \rho\,r
\end{equation}
%
\subsubsection{Simplification: \emph{incompressible fluid} in an
  \emph{isothermal process}}
Momentum balance: $\rho\,\ddot{\Vx} = \div\VT + \rho\,b$
\begin{align}
\nonumber  \div\VT &= \div(p\,\VI + 2\mu\,\VD) \\
  &= -\grad p + 2\mu\div(\half(\grad\dot{\Vx}+\grad{}\tra\dot{\Vx}))\\
  &= -\grad p + \mu\,\div\grad\dot{\Vx}
\end{align}
Fact: $\div\grad{}\tra\dot{\Vx}=\grad\div\dot{\Vx} = 0$ because
$\div\dot{\Vx}=0$ (used above). Hence the balance becomes
%
\begin{equation}
  \rho\,\ddot{\Vx} = -\grad p + \mu\,\div\grad\dot{\Vx} + \rho\,r
\end{equation}
Energy balance: $\div\grad\theta=0$ and $\VD\cdot\VI=0$
\begin{equation}
  \rho\,\dot{\veps} = 2\mu\,\VD\cdot\VD+\rho\,r
\end{equation}
%%%%%%%%%%%%%%%%%%%%%%%%%%%%%%%%%%%
\chapter{Geometrically Linear Thermoelasticity}
%
%
\chapter{Geometrically Linear Viscoelasticity}
%
\chapter{Geometrically Linear Elastoplasticity}
%
\chapter{Appendix}
For a symmetric matrix $\VA$, $(\VA\cdot\VA\tra)=(\VA\cdot\VA)$. This can be used to
exploit the second invariant as
\begin{equation}
  \rtwo_A=\half((\VA\cdot\VI)^2-\VA\tra\cdot\VA) = \half(\UI_A-\VA\cdot\VA)
\end{equation}
Indeed, this has been used while reducing the energy balance for the Newtonian fluid.
%
%
\end{document}
%
